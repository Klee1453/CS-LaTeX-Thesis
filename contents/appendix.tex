\newpage
\addNoindentTocEntry{附录}
\begin{center}
    ~\\[-1.5em]
    \zihao{3}\CJKfamily{fs}\textbf{附\quad 录}
\end{center}

% === 以下内容仅作示例作用,实际内容请根据需要修改 ===

\setcounter{section}{1}
\renewcommand{\thesection}{A\arabic{section}}

\par 以下内容仅作示例作用,实际内容请根据需要修改。

\par \myeqref{equ:sample} 是一个示例公式。

\begin{equation}
  \label{equ:sample}
  A=\overbrace{(a+b+c)+\underbrace{i(d+e+f)}_{\text{虚数}}}^{\text{复数}}
\end{equation}

\par \myfigref{fig:sample} 是一个示例图片。

\begin{figure}[htbp]
  \centering
  \includegraphics[width=.3\linewidth]{images/zju-logo.jpg}
  \caption{\label{fig:sample}示例图片}
\end{figure}

\par \mytabref{tab:sample} 是一个示例表格。

\begin{table}[htbp]
  \caption{\label{tab:sample}自动调节列宽的表格}
  \begin{tabularx}{\linewidth}{c|X<{\centering}}
      \hline
      第一列 & 第二列 \\ \hline
      xxx & xxx \\ \hline
      xxx & xxx \\ \hline
      xxx & xxx \\ \hline
  \end{tabularx}
\end{table}

\par \myalgref{alg:sample} 是一个算法样例。

\begin{algorithm}[H]
  \caption{算法样例}\label{alg:sample}
  \begin{algorithmic}[1] % [1] 参数表示行号从1开始
    \Require 数组 $A[low...high]$ 和起始索引 $low$,结束索引 $high$ % \Require 用于描述输入参数
    \Ensure 排序后的数组 $A$,其中元素按非递减顺序排列 % \Ensure 用于描述输出参数
    \Function{QuickSort}{$A, low, high$}
      \If{$low < high$}
        \State $pivot \gets$ \Call{Partition}{$A, low, high$} % \State 用于普通语句
        \State \Call{QuickSort}{$A, low, pivot-1$}
        \State \Call{QuickSort}{$A, pivot+1, high$}
      \EndIf
    \EndFunction
    
    \Statex
    
    \Function{Partition}{$A, low, high$}
      \State $pivot \gets A[high]$  \Comment{选择最后一个元素作为基准}
      \State $i \gets low-1$  \Comment{小于基准的元素的索引}
      \For{$j \gets low$ to $high-1$}
        \If{$A[j] \leq pivot$}
          \State $i \gets i+1$
          \State 交换 $A[i]$ 和 $A[j]$
        \EndIf
      \EndFor
      \State 交换 $A[i+1]$ 和 $A[high]$
      \State \Return $i+1$ % 返回结果
    \EndFunction
  \end{algorithmic}
\end{algorithm}